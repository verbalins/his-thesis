\documentclass[english]{his-thesis}

% Import user's metadata such as title, name, ISBN, ...
\usepackage{hismetadata}
%% 'Dash ligatures' like '--' do not work correctly in the title.
%% Use '{\textendash}' instead.
\title{Title for your own thesis or proposal}
\subtitle{A subtitle if you want to add that}
\author{Firstname Lastname}
\date{2021}{10}{26} % Dissertation date
\seriesnumber{XX}% in University of Skövde's 'dissertation series'
\isbn{123-12-123456-X-X} % Publication ISBN
\printshop{Stema Specialtryck AB, Borås}

%% \dissertationtype is the official name of the publication's
%% type, either in Swedish or English (whatever applies to you).
%% The value set here is used among others on the title page.
%% Not to be mixed up with \publicationtype
% \dissertationtype{licentiatexamen}
\dissertationtype{licentiatexamen}
\spokenlanguage{engelska}

%% \publicationtype controls whether the document is a
%% dissertation, licentiate thesis, or a proposal. It determines 
%% various formatting aspects, such as the default colors (purple vs. grey).
%% Not to be mixed up with \dissertationtype.
%% Allowed values are: dissertation, licentiate, thesisproposal, or researchproposal.
%\publicationtype{dissertation} or \publicationtype{licentiate} or \publicationtype{thesisproposal} or \publicationtype{researchproposal}
\publicationtype{dissertation}

%% New option, define partner company name and logo.
%% To be used if you are performing the work in collaboration
%% with an external company.
%\partnercompany{Company AB}
%\partnercompanylogo{img/companylogo.png}

%% Abstract shown on the back of the cover and on the spik
\thesissummary{As society is rapidly changed by electrification, digitalization, and globalization, the automotive industry is particularly affected. Reacting to these changes requires predictive tools, the most common being Discrete-Event Simulation combined with Multi-Objective Optimization, referred to as Simulation-Based Optimization. Production lines are most often the focus for optimization while the factory level is ignored even though the factory level has a larger impact on the total production. One reason for not optimizing on the factory level is the time needed due to the complexity of the line models making up the factory model. This licentiate thesis aims to validate and evaluate a model simplification technique to enable optimization of factories leading to cost and resource effective manufacturing.}

%\authorbiography{}

% For spik and presentation:
\decidedby{AMK} % AMK = Avdelningen för Marknad och Kommunikation, Department of Marketing and Communication
\defensedaytimeroom{ons}{26}{oktober}{2021}{10.00}{Portalen, Insikten}
\opponent{Opponent McOpponent, University of Opponents}
\spokenlanguage{english}
% Import functionality for a purple title page
\usepackage{histitlepage}
% Import functionality to set bibliography
\usepackage{hisbibliography}
% Some example bibliography
\addbibresource{example-bibliography.bib}
% Import functionality for text formatting
\usepackage{histextformatting}
% Import functionality for "own publications" page
\usepackage{hisownpublications}


% For generating dummy text.
% Only necessary for the example document, remove for "real dissertations"
\usepackage{lipsum}
\hyphenation{vesti-bulum sce-leris-que}


\begin{document}
% set default font to Georgia 9.5pt
\fontgeorgia{9.5}{11}

\maketitle

\cleardoublepage
\pagenumbering{roman}% Switch to Roman page numbers
\setcounter{page}{1}% Start page numbers from 'i' (Roman 1)
\pagestyle{headings}

\chapter*{Abstract}

\lipsum[1]


\chapter*{Summary}

\lipsum[2-4]


\chapter*{Acknowledgements}

\lipsum[5]

\ownpublications[Some optional text here]{Walley2000,landis1977measureobserveragreement,henkel2006revealingemblinux}{knauth2003prevcompmeasuresshiftwrk,yang2010wikipediacontrib,dedrick2006scope}

\tableofcontents

\listoffigures

\listoftables

\cleardoublepage
\pagenumbering{arabic}% Switch to Arabic page numbers
\setcounter{page}{1}% Start page numbers from '1'
\pagestyle{headings}

\part{The First Part}

\chapter{Introduction}
\label{chap_introduction}

Some \textbf{interesting} \textsf{introduction} and some \emph{interesting} reference.
\par\textbf{Bold} \textit{Italic} \textbf{\textit{BoldItalic}}
\par\textsf{Sans \textbf{Bold} \textit{Italic} \textbf{\textit{BoldItalic}}}
\par\texttt{Mono \textbf{Bold} \textit{Italic} \textbf{\textit{BoldItalic}}}

\begin{equation*}
    \begin{pmatrix}
      1 & 2 & 3  \\
      1 & 2 & 3  \\
      1 & 2 & 3
    \end{pmatrix}
    = \left( \sum_{i=1}^{n-1} i \right)
    = \left( n \right)
\end{equation*}

\textcite{Walley2000} writes that \LaTeX\ is really good.
\textcite{entrywithurl} agrees.

\begin{table}
\begin{tabular}{L{.6\linewidth}L{.3\linewidth}}%
\toprule%
\lstinline|\cite{Walley2000}| & \cite{Walley2000} \\[0.5em]%
\lstinline|\cite{Walley2000,knauth2003prevcompmeasuresshiftwrk,yang2010wikipediacontrib}| & \cite{Walley2000,knauth2003prevcompmeasuresshiftwrk,yang2010wikipediacontrib} \\
\midrule%
\lstinline|\parencite{Walley2000}| & \parencite{Walley2000} \\[0.5em]%
\lstinline|\parencite{Walley2000,knauth2003prevcompmeasuresshiftwrk,yang2010wikipediacontrib}| & \parencite{Walley2000,knauth2003prevcompmeasuresshiftwrk,yang2010wikipediacontrib} \\
\midrule%
\lstinline|\autocite{Walley2000}| & \autocite{Walley2000} \\[0.5em]%
\lstinline|\autocite{Walley2000,knauth2003prevcompmeasuresshiftwrk,yang2010wikipediacontrib}| & \autocite{Walley2000,knauth2003prevcompmeasuresshiftwrk,yang2010wikipediacontrib} \\
\midrule%
\lstinline|\textcite{Walley2000}| & \textcite{Walley2000} \\[0.5em]%
\lstinline|\textcite{Walley2000,knauth2003prevcompmeasuresshiftwrk,yang2010wikipediacontrib}| & \textcite{Walley2000,knauth2003prevcompmeasuresshiftwrk,yang2010wikipediacontrib} \\
\bottomrule%
\end{tabular}%
\caption{Various literature citations.}
\end{table}

\begin{figure}
\centering
A figure with a long caption.
\par
\caption{Sed commodo posuere pede. Mauris ut est. Ut quis purus. Sed ac odio.
Sed vehicula hendrerit sem. Duis non odio. Morbi ut dui. Sed accumsan risus eget odio.
In hac habitasse platea dictumst. Pellentesque non elit. Fusce sed justo eu urna porta tincidunt.
Mauris felis odio, sollicitudin sed, volutpat a, ornare ac, erat.
Morbi quis dolor.
Donec pellentesque, erat ac sagittis semper, nunc dui lobortis purus, quis congue purus metus ultricies tellus.
Proin et quam.
Class aptent taciti sociosqu ad litora torquent per conubia nostra, per inceptos hymenaeos.
Praesent sapien turpis, fermentum vel, eleifend faucibus, vehicula eu, lacus.%
}
\end{figure}

\lipsum[6-10]

\section{Outline}
\label{sec_outline}

\lipsum[11]

\part{A New Part}

\chapter{Some Interesting New Method}

\lipsum[13-14]

\section{The Method}

\lipsum[1]

\begin{equation}
\label{eq_the_method}
    f(x) = x_1^2 + x + 1
\end{equation}

\lipsum[2]

\subsection{More Details}

\cite{krishnamurthy2003manageroverview,dedrick2006scope,henkel2006revealingemblinux,fitzgerald2003trencheslessons}
\lipsum[15-19]

\chapter{Summary and Conclusions}
\label{chap_conclusions}

\lipsum[20-25]

\begin{appendix}

\chapter{How I Got My Appendix Removed}

\lipsum[23-42]

\end{appendix}

\begin{fullarticles}
	\fullarticle[scale=.7,trim={0mm 5mm 0mm 5mm},clip=true,pages=-]{fischer2017doctoraldissertationmanual}{manual.pdf}[\href{https://creativecommons.org/licenses/by-nc-nd/4.0/}{\ccbyncnd}]
	\fullarticle[scale=.7,trim={0mm 5mm 0mm 5mm},clip=true,pages=1]{fischer2017doctoraldissertationmanual}{manual.pdf}[Reprinted from][with permission from IOS Press. The publication is available at IOS Press through \href{http://dx.doi.org/10.3233/978-1-61499-902-7-467}{http://dx.doi.org/10.3233/978-1-61499-902-7-467}.]
\end{fullarticles}

% from package 'hisbibliography'
\listofreferences

% from package 'hisbibliography'
\dissertationlist

\end{document}
