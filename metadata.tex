%% 'Dash ligatures' like '--' do not work correctly in the title.
%% Use '{\textendash}' instead.
\title{Title for your own thesis or proposal}
\subtitle{A subtitle if you want to add that}
\author{Firstname Lastname}
\date{2021}{10}{26} % Dissertation date
\seriesnumber{XX}% in University of Skövde's 'dissertation series'
\isbn{123-12-123456-X-X} % Publication ISBN
\printshop{Stema Specialtryck AB, Borås}

%% \dissertationtype is the official name of the publication's
%% type, either in Swedish or English (whatever applies to you).
%% The value set here is used among others on the title page.
%% Not to be mixed up with \publicationtype
% \dissertationtype{licentiatexamen}
\dissertationtype{licentiatexamen}
\spokenlanguage{engelska}

%% \publicationtype controls whether the document is a
%% dissertation, licentiate thesis, or a proposal. It determines 
%% various formatting aspects, such as the default colors (purple vs. grey).
%% Not to be mixed up with \dissertationtype.
%% Allowed values are: dissertation, licentiate, thesisproposal, or researchproposal.
%\publicationtype{dissertation} or \publicationtype{licentiate} or \publicationtype{thesisproposal} or \publicationtype{researchproposal}
\publicationtype{dissertation}

%% New option, define partner company name and logo.
%% To be used if you are performing the work in collaboration
%% with an external company.
%\partnercompany{Company AB}
%\partnercompanylogo{img/companylogo.png}

%% New options, the path to an image on the front cover, and an image of you on the back cover.
\coverimage{template/Manuscript/foto.jpg}
\coverphoto{template/Manuscript/face.jpg}

%% Abstract shown on the back of the cover and on the spik
\thesissummary{As society is rapidly changed by electrification, digitalization, and globalization, the automotive industry is particularly affected. Reacting to these changes requires predictive tools, the most common being Discrete-Event Simulation combined with Multi-Objective Optimization, referred to as Simulation-Based Optimization. Production lines are most often the focus for optimization while the factory level is ignored even though the factory level has a larger impact on the total production. One reason for not optimizing on the factory level is the time needed due to the complexity of the line models making up the factory model. This licentiate thesis aims to validate and evaluate a model simplification technique to enable optimization of factories leading to cost and resource effective manufacturing.}

% For spik and presentation:
\decidedby{AMK} % AMK = Avdelningen för Marknad och Kommunikation, Department of Marketing and Communication
\defensedaytimeroom{ons}{26}{oktober}{2021}{10.00}{Portalen, Insikten}
\opponent{Opponent McOpponent, University of Opponents}
\spokenlanguage{english}

